% LuaLaTeX

\documentclass[a4paper, twoside, 10pt]{article}

\usepackage[latin]{babel}

\usepackage{geometry}
%\geometry{papersize={7.444in,9.681in},total={4.8in,6.8in}}
\geometry{papersize={10.5cm,17.3cm},left=1cm,right=1cm,top=1.2cm,bottom=1.2cm}

\usepackage{fontspec}
\setmainfont[Ligatures={Common, TeX}]{Junicode}

\usepackage{multicol}
\usepackage{color}
\usepackage{lettrine}
\usepackage{fancyhdr}

\usepackage{ecclesiastic}

\newcommand{\rubricatum}[1]{\textcolor{red}{#1}}

% kalendarium
\newenvironment{kalendarium}{\setlength{\parindent}{0cm}}{}
\newcommand{\kMensis}[1]{
  \vspace{2mm}
  \hfill {\large\rubricatum{#1}} \hfill
  \vspace{1mm}}
\newcommand{\kDies}[3]{#1 \hspace{2mm} #2 \rubricatum{#3}

}

\newcommand{\titulusTomus}[1]{
  \begin{center}
    {\Huge
      Officia propria

      Provinciæ Pragensis}

    {\Large
      a S. Sede approbata et indulta}

    \vfill

    #1

    \vfill

    Editio Sancti Wolfgangi
  \end{center}
}

\begin{document}

\pagestyle{empty}

\titulusTomus{tomus prior}

\cleardoublepage

\pagestyle{plain}

\section*{Kalendarium perpetuum\\Provinciæ Pragensis}

\begin{kalendarium}
% 1928, aestiva, without commemorations and suppressed octaves
\kMensis{Novembris}

\end{kalendarium}

\cleardoublepage

\begin{multicols}{2}

  Die 1 Decembris

\rubricatum{In archidioecesi Pragensi}

S. Edmundi Campion

\rubricatum{Mart. e S. J.}

\rubricatum{Oratio}

Deus, qui verae fídei et Sedis Apostólicae primátui pro\-pug\-nán\-do
beátum Mártyrem tuum Edmúndum invícta fortitúdine roborásti:
ejus précibus exorátus,
nostrae, quaé\-su\-mus, infirmitáti succúrre;
ut fortes in fide adversário resístere usque in finem valeámus.
Per Dóminum.


\end{multicols}

\cleardoublepage

\pagestyle{empty}

\titulusTomus{tomus alter}

\cleardoublepage

\pagestyle{plain}

\section*{Kalendarium perpetuum\\Provinciæ Pragensis}

\begin{kalendarium}
% TODO - 1962 reformed feast ranks
% TODO - commemorationes?

% 1928, aestiva, without commemorations and suppressed octaves
\kMensis{Majus}

\kDies{17}{S. Paschalis Baylon Conf.}{duplex}
\kDies{18}{S. Venantii Mart.}{duplex}
\kDies{19}{S. Petri Cælestini Papæ et Conf.}{duplex}
\kDies{20}{S. Bernardini Senensis Conf.}{semiduplex}
\kDies{25}{S. Gregorii VII. Papæ et Conf.}{duplex}
\kDies{26}{S. Philippi Nerii Conf.}{duplex}
\kDies{27}{S. Bedæ Venerabilis Conf. et Eccl. Doct.}{duplex}

\end{kalendarium}


\end{document}
